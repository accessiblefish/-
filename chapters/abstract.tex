%% This is file 'abstract.tex'

\begin{abstract}
	\linespread{1.5}
由于泥沙与水流的相互作用,使得河流发生演变,因此泥沙特性与水流特性均是河流动力学的重要研究课题。当水流中含有植物时,水流的紊动特性会发生明显的改变,从而引起泥沙的一些特性如沉速发生改变。本文以实验为基础,结合理论分析,研究了在静水条件下刚性植物对泥沙沉速的影响,同时在水槽中通过改变流量来研究在恒定均匀流条件下非淹没植物对泥沙沉降轨迹的影响,得到如下主要结论:

此处填写中文摘要

% 中文关键词使用中文;隔开
\keywords{关键词1;关键词2;关键词3}
\end{abstract}


\begin{enabstract}
	\linespread{1.5}
Fluvial river processes evolve over time in response to the constant interaction between sediment and the water column. If vegetation is present within the water column, the change in turbulence characteristics will impact the movement of sediment, in particular the settling velocity. In this paper, the influence of vegetation on the settling velocities of sediment particles is studied experimentally. The  non-submerged vegetation friction factor in steady uniform flow is considered by under different flume discharge quantities. The main outcomes can be summarized as follows:

此处填写英文摘要

% 英文关键词使用英文;隔开
\enkeywords{sediment; rigid vegetation; settling velocity; turbulence characterize}

\end{enabstract}
