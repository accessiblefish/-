%% This is file 'chapter2.tex'

\chapter{基于蒙特卡洛的}
\label{chap:inverseproblem}

\section{引言}
引言……

……

\section{xxxxxxx影响因素分析}

\subsection{xxx细分}

……概率密度函数为:
\begin{equation}
\label{eq:kduiuijm}
    f_s(t_1)=\left\{
    \begin{array}{lr}
    \frac{1}{\sigma_s\sqrt{2\pi}}e^{-\frac{(t_1-\mu_s)^2}{2\sigma_s^2}}, & \mu_S-12<t_1\leq 24\\
    \frac{1}{\sigma_s\sqrt{2\pi}}e^{-\frac{(t_1+24-\mu_s)^2}{2\sigma_s^2}}, & 0<t_1\leq \mu_s-12
    \end{array}
    \right.
\end{equation}
\noindent 式中:$t_1$为开始充电时间;$\mu_S$为函数$f_s(t_1)$中$t_1$的期望;$\sigma_S$为$f_s(t_1)$中$t_1$的标准差。

……如表\ref{tab:parameter}和表\ref{tab:gdlv}所示:

\begin{table}[H]\small
    \centering
    \caption{不同类型电动汽车日行驶里程对数正态分布参数}
    \label{tab:parameter}
    \begin{tabularx}{\textwidth}{@{} >{\centering\arraybackslash}X >{\centering\arraybackslash}X >{\centering\arraybackslash}X >{\centering\arraybackslash}X @{}}
    \toprule
	参数 & 私家车 & 公交车 & 出租车\\\midrule
    $\mu_D$ & 3.2 & 3.0 & 5.1\\
    $\sigma_D$ & 0.88 & 0.8 & 0.3\\
    \bottomrule
    \end{tabularx}%
\end{table}

\begin{table}[H]\small
    \centering
    \caption{不同电动汽车开始充电时间和概率}
    \label{tab:gdlv}
    \begin{tabularx}{\textwidth}{@{} >{\centering\arraybackslash}X >{\centering\arraybackslash}X >{\centering\arraybackslash}X >{\centering\arraybackslash}X \>{\centering\arraybackslash}X @{}}
    \toprule
	{电动汽车类型 & 功能区 & 充电时段 & 充电概率 & 充电开始时间分布}\\\midrule
    \multirow{2}{*}{出租车} & \multirow{2}{*}{商业区} & 02:00$\sim$04:00 & 100\% & \multirow{2}{*}{均匀分布}\\
    \cline{3-4}
    & & 11:30$\sim$4:00 & 100\% &\\\midrule
    公交车 & 工作区 & 18:00$\sim$08:00 & 100\% & 均匀分布\\\midrule \multirow{2}{*}{私家车} & 工作区 & 08:00$\sim$17:00 & 20\% & $N(8.9,1.5^2)$\\\cline{2-5}
    & 居民区 & 17:00$\sim$07:00 & 80\% & $N(17.6,3.4^2)$\\
    \bottomrule
    \end{tabularx}
\end{table}


\section{本章小结}
